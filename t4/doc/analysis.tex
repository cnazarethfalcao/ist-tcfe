\section{Theoretical Analysis}
\label{sec:analysis}
\par The next subsections explain the theoretical analysis made to predict the output voltages and impedances in both the Gain and Ouput stages, designed by the teacher previously to the lab.

\subsection{Gain Stage}

Table\ref{tab:dadosg} shows the important values used in the circuit for the gain stage.

\begin{table}[h]
  \centering
  \begin{tabular}{|l|r|}
    \hline    
    {\bf Name} & {\bf Value}\\ \hline
    $V_{T}$ & 2.500000e-02 \\ \hline 
$Vcc$ & 1.200000e+01 \\ \hline 
$Beta$ & 1.787000e+02 \\ \hline 
$V_{A}$ & 6.970000e+01 \\ \hline 
$R_{E}$ & 1.000000e+02 \\ \hline 
$R_{C}$ & 3.500000e+03 \\ \hline 
$R_{bias1}$ & 2.300000e+04 \\ \hline 
$R_{bias2}$ & 2.000000e+03 \\ \hline 
$C$ & 4.500000e-04 \\ \hline 
$V_{BEON}$ & 7.000000e-01 \\ \hline 
$R_{S}$ & 1.000000e+02 \\ \hline 

  \end{tabular}
  \caption{Gain Stage - Parameters.}
  \label{tab:dadosg}
\end{table}

\par The Gain stage circuit containes a coupling capacitor that works as a DC block, separating the DC component of the base of the transistor and the DC component of the input (which is approximately zero). This capacitor blocks some low frequencies. A transistor and resistances were used and also a bypass capacitor. While stabilizing the temperature, resistance $R_{E}$ lowers the gain so it was necessary to improve it. To do so, this capacitor was used, with the objective of bypassing $R_{E}$ in order to improve the gain at medium frequencies. This capacitor works as an open-circuit for low frequencies(DC) and as short-circuit for medium frequencies(AC).

\par First, an operating point analysis was made using the DC model. With the mesh analysis the needed voltages and currents were determined. It also allowed to confrm if the transistor was forwardly biased.

\par Then, an incremental analysis was made in order to determine the impedances and the incremental circuit gain.

\par The obtained results are given in Table\ref{tab:gr}.

\begin{table}[h]
  \centering
  \begin{tabular}{|l|r|}
    \hline    
    {\bf Name} & {\bf Value}\\ \hline
    $Input impedance$ & 9.359215e+02 \\ \hline 
$Output impedance$ & 3.131224e+03 \\ \hline 
$Gain$ & 2.653995e+02 \\ \hline 

  \end{tabular}
  \caption{Gain Stage - Results.}
  \label{tab:gr}
\end{table}


\par Taking a closer look at the results given by the Table, one sees that the output impedance of the gain stage has a big value so the circuit cannot be connected to an $8 Ohm$ load. This is the reason why an output stage is necessary.

\newpage

\subsection{Output Stage}
\par As previously refered, the output stage is needed to provide a low impedance to the load. The chosen parameters for this circuit are shown in Table\ref{tab:dadoso}.

\begin{table}[h]
  \centering
  \begin{tabular}{|l|r|}
    \hline    
    {\bf Name} & {\bf Value}\\ \hline
    $Beta$ & 2.273000e+02 \\ \hline 
$V_{A}$ & 3.720000e+01 \\ \hline 
$R_{E}$ & 2.000000e+02 \\ \hline 
$V_{BE ON}$ & 7.000000e-01 \\ \hline 
$V_{in}$ & 3.791166e+00 \\ \hline 

  \end{tabular}
  \caption{Output Stage - Parameters.}
  \label{tab:dadoso}
\end{table}


\par Similar to before, operating point and incremental analysis were produced to obtain the output impedance and gain (which was supposed to be near one).

\par The obtained results are shown in Table\ref{tab:or}.

\begin{table}[h]
  \centering
  \begin{tabular}{|l|r|}
    \hline    
    {\bf Name} & {\bf Value}\\ \hline
    $Input impedance$ & 1.821626e+04 \\ \hline 
$Output impedance$ & 6.632304e-01 \\ \hline 
$Gain$ & 9.960174e-01 \\ \hline 

  \end{tabular}
  \caption{Output Stage - Results.}
  \label{tab:or}
\end{table}

\par As one can see in the results, the gain for the output stage is nearly one, which was the objetive and the output impedance is very small, making it perfect to connect to the load.

\newpage

\subsection{Total Results}

\par In this lab assignement, it was also required to analyse the frequency response of the gain and compute into a graphic. This analysis is shown in Figure\ref{fig:freqresp}.

\begin{figure}[h] \centering
\includegraphics[width=0.6\linewidth]{freqgain.eps}
\caption{Frequency response of the gain .}
\label{fig:freqresp}
\end{figure}


\par Considering the total circuit, the results obtained for the most important quantities are shown in Table\ref{tab:totalrmat}.

\begin{table}[h]
  \centering
  \begin{tabular}{|l|r|}
    \hline    
    {\bf Name} & {\bf Value}\\ \hline
    $Input impedance$ & 9.359215e+02 \\ \hline 
$Output impedance$ & 1.323807e+01 \\ \hline 
$Gain$ & 2.643425e+02 \\ \hline 
$Lower Cut-off Frequency$ & 3.125716e+01 \\ \hline 
$Bandwidth$ & 1.434561e+06 \\ \hline 
$Cost$ & 1.229100e+03 \\ \hline 
$Merit$ & 9.870730e+03 \\ \hline 

  \end{tabular}
  \caption{Total Circuit - Results.}
  \label{tab:totalrmat}
\end{table}


\par Observing these results, one can see that the two stages can be connected without significant signal loss because the output impedance of the gain stage is much smaller than the input impedance of the output stage and the values are quite compatible.

\par Looking at the values for the output impedance of the circuit, one notices that its value is small and good to connect to the $8 Ohm$ load.

\newpage
