\section{Theoretical Analysis}
\label{sec:analysis}
\par The next subsections explain the theoretical analysis made to predict the voltage in both the envelope detector and voltage regulator. Table\ref{tab:dados} shows the important values used in the circuit.


\begin{table}[h]
  \centering
  \begin{tabular}{|l|r|}
    \hline    
    {\bf Name} & {\bf Value}\\ \hline
    $n(transformer)$ & 4.472136e+00 \\ \hline 
$R_{envelope}$ & 8.000000e+04 \\ \hline 
$C_{envelope}$ & 8.500000e-05 \\ \hline 
$n_{diodes}$ & 1.800000e+01 \\ \hline 
$R_{vr}$ & 2.810000e+04 \\ \hline 

  \end{tabular}
  \caption{Chosen parameters.}
  \label{tab:dados}
\end{table}


\par At first, a transformer with a ratio n:1 was used, leading to the following relation.

\begin{equation}
  A_{secondary} = \frac {A_{primary}{n}}.
  \label{eq:transformer}
\end{equation}

\subsection{Envelope Detector}
\par The input voltage in the envelope detector will be $V_secondary$, given by:

\begin{equation}
  V_{secondary} = A_{secondary}*cos(w*t).
  \label{eq:transformer}
\end{equation}

With w = 2$\pi$f.

\par Then, an envelope detector was produced, using a full-wave rectifier (four diodes and a resistor) and a capacitor, as seen in Figure\ref{fig:circuit}.
 The output voltage in the envelope detector will depend if the diodes are on or off. 
\par When the diodes are on, assuming that the diodes are linear components, the voltage in the envelope is equal to the voltage if only a rectifier existed, which means that it will be equal to the abslotute value of $V_{secondary}$. 

\begin{equation}
  V_{envelope} = |V_{secondary}|.
  \label{eq:veon}
\end{equation}

When the diodes are off the voltage in the envelope will be given by:

\begin{equation}
  V_{envelope} = A_{secondary}*|cos(w*t_{off})|e^(-\frac{t-t_{off}}{R_{envelope}*C}).
  \label{eq:veoff}
\end{equation}

\par Because the voltage is given by a sinusoidal fuction that has a period, in this case, since we used a full bridge rectifier, the time between two consecutive $t_{off}$ will be T/2 and the same applies for $t_{on}$. This relation enabled us to determine the output voltage in the envelope detector.
\par To determine $t_{off}$ we used the following expression:

\begin{equation}
  t_{off} = \frac{1}{w}*arctg(\frac{1}{w*R_{envelope}*C})
  \label{eq:toff}
\end{equation}

\par The value of $t_{on}$ is the time were the curves of the rectified signal and the voltage in the envelope when the diodes are off intersect.

\par Because the voltage is given by a sinusoidal fuction that has a period, in this case, since we used a full bridge rectifier, the time between two consecutive $t_{off}$ will be (T/2) and the same applies for $t_{on}$. This relation enabled us to determine the output voltage in the envelope detector for every value of $t$. The obtained results are shown in Figure\ref{fig:venvelope}.

\begin{figure}[h] \centering
\includegraphics[width=0.6\linewidth]{envelope.eps}
\caption{Output Voltage in the Envelope Detector[v].}
\label{fig:venvelope}
\end{figure}

\newpage

\subsection{Voltage Regulator}

\par The voltage regulator used in this circuit consists in a resistor in series with a limiter circuit (positive voltage limiter) that uses a series of diodes with $n_{diodes}$. The objective of this part of the circuit is to achieve the required DC voltage of 12V as it was stated before.
\par In order to predict the final result, an incremental analysis was used, which means that:

\begin{equation}
  V_{out} = V_{o} + v_{o}  
  \label{eq:Vout}
\end{equation}

\par Since $V_{envelope}$ is greater than 12V, $V_{o} = V_{envelope}$. The incremental part is given by the following expression:

\begin{equation}
  v_{o} = \frac{n_{diodes}*r_{d}}{n_{diodes}*r_{d} + R_{vr}}*(V_{envelope} - avg(V_{envelope}))
  \label{eq:vo}
\end{equation}

\par With,

\begin{equation}
  r_{d} = \frac{\eta *V_{T}}{I_{s}*e^(\frac{12}{n_{diodes}*\eta *V_{T}}})
  \label{eq:rd}
\end{equation}

And $V_{T} = 0,025V$ (thermal voltage) , $I_{s} = 1*10^(-14)$ (reverse saturation current) and $\eta = 1$ (material constant).

\par With the chosen parameters of Table\ref{tab:dados}, the obtained values and graphics are shown in the next figures.

\begin{figure}[h] \centering
\includegraphics[width=0.6\linewidth]{vo.eps}
\caption{Output Voltage [V].}
\label{fig:vout}
\end{figure}

\newpage

\begin{figure}[h] \centering
\includegraphics[width=0.6\linewidth]{dev.eps}
\caption{Deviation ($v_{out}$ - 12) [V].}
\label{fig:deviation}
\end{figure}


\par Finally, the important obtained results are shown in the following table:

\begin{table}[h]
  \centering
  \begin{tabular}{|l|r|}
    \hline    
    {\bf Name} & {\bf Value}\\ \hline
    $average$ & 1.200000e+01 \\ \hline 
$deviation$ & 1.776357e-15 \\ \hline 
$ripple$ & 3.104277e-04 \\ \hline 
$cost$ & 1.953000e+02 \\ \hline 
$merit$ & 1.644147e+01 \\ \hline 

  \end{tabular}
  \caption{Theoretical Results.}
  \label{tab:tr}
\end{table}

\newpage


